\documentclass{report}

\title{Designing a Search for the BBC}

\begin{document}

\chapter{The Problem}

\section{Introduction}

The BBC has been publishing content on the World Wide Web since the
mid 1990s and since then the amount and the diversity has increased
exponentially. Large websites or indeed the web as a whole would not have
been usable nor useful without the rise in quality of web search engines to
help people find content based on keywords or phases that describe or
appear in the sought after pages.

A large challenge for any web search application is to provide a common
interface and set of user interactions that can equally index, search
and link to a diverse range of types of information -- be it in the form
of text, images, video or games. A more recent challenge has been to
achieve this in a near-real time way to catch up with the rapid rate at
which content is added to the web (particulary from microblogging websites
such as Twitter).

Whilst there are successful general web search engines such a Google that
will find any piece of content, there is a perceived need for a
BBC-specific search that makes better use of the internal knowledge the BBC
has about its own content.

\section{Objectives}

The purpose of this report is to start the high-level design -- and perhaps
look at the lower levels of one or two aspects -- of a search application
specific to BBC content. The intent is to provide a consistent user interface
that allows the audience to find equally news articles, sport results,
TV catch-up, education resources and everything else the BBC produces.

\chapter{Design}

\chapter{Analysis}

\chapter{Evaluation}

\end{document}
