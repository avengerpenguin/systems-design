\documentclass[a4paper]{report}

\usepackage{graphicx}
\usepackage{rotating}
\usepackage{verbatim}

\title{Designing a Search for the BBC}
\author{Ross Fenning}

\begin{document}

\maketitle

\tableofcontents

\chapter{The Problem}

\section{Rationale}

The BBC has been publishing content on the World Wide Web since the
mid 1990s and since then the amount and the diversity has increased
exponentially. Large websites or indeed the web as a whole would not have
been usable nor useful without the rise in quality of web search engines to
help people find content based on keywords or phases that describe or
appear in the sought after pages.

A large challenge for any web search application is to provide a common
interface and set of user interactions that can equally index, search
and link to a diverse range of types of information -- be it in the form
of text, images, video or games. A more recent challenge has been to
achieve this in a near-real time way to catch up with the rapid rate at
which content is added to the web (particulary from microblogging websites
such as Twitter).

Whilst there are successful general web search engines such a Google that
will find any piece of content, there is a perceived need for a
BBC-specific search that makes better use of the internal knowledge the BBC
has about its own content.

\section{Objectives}

The purpose of this report is to start the high-level design -- and perhaps
look at the lower levels of one or two aspects -- of a search application
specific to BBC content. The intent is to provide a consistent user interface
that allows the audience to find equally news articles, sport results,
TV catch-up, education resources and everything else the BBC produces.

The target audience for the BBC is effectively the entire population of the UK
and amongst those that do make use of BBC services, there is much diversity
of needs, preferences and technical ability. It is clear it is no small
task to design a search-based discovery mechanism of millions of diverse
pieces of content aimed at millions of diverse people.

Using \emph{Soft Systems Methodology} (SSM)
\cite{checkland2006learning}, we can stand back from
an ontological approach of defining what the search system \emph{is} or
\emph{comprises} and instead take an \emph{epistemological} view of search
as a system. With this view, we could consider a system that holistically
transforms members of the public's desires to find online content into
the consumption of that content -- whether those desires are \emph{precise}
(e.g. they want an exact article known by headline they saw earlier or a
particular programme they missed on television) or those desires are
\emph{fuzzy} (e.g. news about a certain topic, any comedy programme, learning
materials about the Industrial Revolution).

Checkland \cite{checkland1990soft} decribed a \emph{Rich Picture} approach to
representing a problem situation early in SSM approaches.
Given the size and complexity of the
search system as a whole, a useful initial step is to create such an informal
representation of what is known about the problem. Figure~\ref{rich-picture}
shows what I know of the audience, search and most BBC online content areas.
Note that not all areas are covered and a strong emphasis is placed on TV
catch-up (e.g. via the iPlayer product). Radio catch-up is not mentioned
as it shares a lot of similarity with television in terms of use and any
differences are out of scope for this design.

The overall design objective will be to create an initial proposal for a
search application to drive the missing components within the holistic
system depicted in figure~\ref{rich-picture}. Some subsystems already
exist, e.g. for journalists to write news articles and publish them on the
BBC News website, but for the purposes of this design exercise, we will
assume no existing application to drive the search and discovery of
those websites.

It is likely that the design will look to integrate with
existing subsystems where possible rather than attempt to replicate
work already done. For example, journalists will prefer that a search
application can integrate with the system into which they are publishing
their articles instead of being required to publish their articles into
two systems.

\chapter{Design}

\section{Use cases}

From the rich picture in figure~\ref{rich-picture}, we can attempt to
extract the activities that are clearly within the remit of a search
application. For example, the ability for editorial staff to manage
the content of the search indexes sounds like a feature the search
application would provide. Conversely, television actors and other
contributors to a programme are likely to interact only within a television
production subsystem with a producer or content editor being responsible
for publishing information about the final production's broadcast and
availability for streaming online.

This is not to say that we can simply cross off certain elements from
our depiction of the problem because they do not directly interact
with the subsystem being designed. The \emph{systems thinking} approach
advocated by Checkland \cite{checkland1999systems} encourages us to
consider the irreducible properties of each system at each level
of abstraction. Thus we need to consider not only a search application
subsystem that solves specific problems for its immediate users, but
also an application that contributes to the desirable, emergent
properties of the BBC service as a whole.

In the specific example of television programming, we need to maintain
systems thinking throughout the design process to ensure that we
create a search application that both meets the needs of the public
using the application to search for programmes and forms part of
a television production and delivery system that itself meets the
needs of the television-watching public.

Thus a suitable design strategy is to apply systems design to
the search application in isolation -- as a \emph{hard problem} --
but then to use the wider system to inform, shape and evaluate that design.

Having extracted the activities that appear to pertain to the
search application directly (and the actors involved in those activities),
we can produce a use case UML diagram like that in figure~\ref{use-case}.

This illustrates only a subset of the expected behaviours for a full
BBC search application, but touches on some of the diversity of the
potential uses. For the purposes of our initial design, we can next look
into defining the system behaviour for some of these use cases.

\begin{sidewaysfigure}
  \begin{center}
    \includegraphics[width=\linewidth]{use_case.png}
  \end{center}
  \caption{Use case diagram for BBC Search application\label{use-case}}
\end{sidewaysfigure}

\section{System Behaviour}

\begin{comment}
@startuml sequence_producer.png

skinparam monochrome true

actor Producer
participant "Programmes Database" as prog
participant "Search Index Service" as service
participant "Search Indexes" as indexes

Producer ->> prog

activate prog
prog ->> service
deactivate prog

activate service
service ->> indexes
deactivate service

activate indexes
indexes -> indexes : reindex
deactivate indexes

@enduml
\end{comment}
\begin{figure}[t]
  \begin{center}
    \includegraphics[width=\linewidth]{sequence_producer.png}
  \end{center}
  \caption{Sequence diagram showing publication of programme information to the search indexes\label{sequence-producer}}
\end{figure}

Figure~\ref{sequence-user} shows a sequence diagram defining a user
interacting with the search system and figure~\ref{sequence-producer}
shows how a TV producer indirectly interacts with search by providing
information that ultimately ends up in the search indexes.

The key design decision in figure~\ref{sequence-producer} is the use of
asynchronous messages only. A non-blocking set of interactions such as
publish-subscribe \cite{hohpe2004enterprise} is a good way to decouple
systems that produce and store progamme information from the search
application systems. If the systems surrounding the programmes database
can be built a \emph{channel adapter} \cite{hohpe2004enterprise} to integrate
it to a messaging system, then the search indexes can receive changes to
information without actors such as TV producers, journalists and content
editors even being aware this is happening.

\begin{comment}
@startuml sequence_user.png

skinparam monochrome true

actor User
participant "BBC Website" as www
participant "Search Service" as search
participant "Search Indexes" as indexes
participant "Source Service" as source

User -> www : types query into search box
activate www

  www -> search
  activate search

    note over search
      The service may want to vary the
      query for a number of reasons such
      as the type of device the user has
      or the nature of the query itself.
    end note

    search -> search : decide query parameters

    search -> indexes : raw query
    activate indexes
    indexes --> search : list of matching items
    deactivate indexes

    loop for each item
      search -> source : Fetch more info about item
      activate source

      note right of search
        We can enrich the retrieved items with
        more information from the original source
        systems rather than replicate all domain
        knowledge in the search indexes.
      end note

      source --> search : Returns richer info
      deactivate source
    end

  search --> www : list of enriched results
  deactivate search

www --> User : search results page
deactivate www

@enduml
\end{comment}
\begin{figure}[p]
  \begin{center}
    \includegraphics[width=\linewidth]{sequence_user.png}
  \end{center}
  \caption{Sequence diagram showing a user interacting with search\label{sequence-user}}
\end{figure}

Note in figure~\ref{sequence-user} that while the search indexes
will contain representations of content
from several source systems, the intention shown is that richer information
about the domain model will not be held in the search indexes. This goes
along with the principle of separation of concerns\cite{dijkstra1982role}
in that the search index component can focus on optimising its data
structures around retrieval. This kind of modularity also allows source
systems maintained by other teams to take the responsibility of the
accuracy of the information, which fits in with the wider holistic view
of the system in that the volume of information involved requires that
separate teams are responsible for the accuracy of their data.

This can be seen as similar to the \emph{Lazy Load} pattern
\cite{fowler2002patterns} in that the search indexes will only return
stub objects that are capable of retrieving the fuller information
to need. This, however, could lead to a lot of calls to different
service applications per page of results. This can be done in parallel, but
the fact still remains that the user has to wait for all this to
assemble before seeing even one result.

One solution to this in modern web application design is to
push the lazy loading into the web browser using AJAX. \cite{garrett2005ajax}
Such a solution is depicted in figure~\ref{sequence-ajax}. Whilst
this still requires just as many calls to backend services --
perhaps even more complexity as the calls are going through
more layers -- it can give a user experience than appears more responsive.

Preparing a search results page with minimal information that is then
augmented asynchronously is a user experience technique that gives the
illusion of lower latency; the additional information can update the page
during the user's reaction time in the best case. This is an important
aspect of user experience to get right as multivariate testing at Google
in 2006 showed that an additional 500ms of latency dropped traffic
and revenue by 20\%. \cite{linden2006marissa}

\begin{comment}
@startuml sequence_ajax.png

skinparam monochrome true

actor User
participant "BBC Website" as www
participant "Search Service" as search
participant "Search Indexes" as indexes
participant "Source Service" as source

User -> www : types query into search box
activate www

  www -> search
  activate search

    note over search
      The service may want to vary the
      query for a number of reasons such
      as the type of device the user has
      or the nature of the query itself.
    end note

    search -> search : decide query parameters

    search -> indexes : raw query
    activate indexes
    indexes --> search : list of matching items
    deactivate indexes

    search --> www : Return stub items
    deactivate search
    
    www --> User : search results page

    note over www
      Website remains active using Javascript to do 
      AJAX requests for further information for
      each result.
    end note

    loop for each item
      www ->> search : Fetch more info about item
      activate search
      search -> source : Proxy to source for info
      activate source
      source --> search : Return info in source's own format
      deactivate source
      search --> www : Return info in common format
      deactivate search
      www --> User : Update page in front of user as info arrives
    end

deactivate www

@enduml
\end{comment}
\begin{figure}[p]
  \begin{center}
    \includegraphics[width=\linewidth]{sequence_ajax.png}
  \end{center}
  \caption{Sequence diagram showing rich information lazy-loaded via AJAX\label{sequence-ajax}}
\end{figure}

\section{Domain Model}

It has been expressed already that a BBC-wide search application
is going to have to index, retrieve and display a diverse set of
information. Parts of the application might well want to incorporate
a domain model \cite{fowler2002patterns} so as to understand how
to perform each of these actions against each possible item
the search application could return as a search result. A maximal
model is shown in figure~\ref{model} that attempts to capture
a good proportion of the content and concepts the BBC has
been attempting to model over several years. This model is
an aggregate of individual ontologies developed for specific purposes,
but given the search application has to provide the discovery for
the full set of this information, it is not unreasonable that
a search application would have a domain model that covers the totality.

Note that complexity of information for programmes alone. \cite{raimond2009bbc}
A programme to a member of the public could actually refer to an
exact episode or indeed the \emph{brand}, i.e. the title that
persists to all episodes. An example of a brand would be
\emph{Doctor Who}, which itself contains multiple series, which in turn
contain a collection of episodes. Note that the brand itself has episodes
as immediate children that do not live under a series, e.g. Christmas specials.
It is also the case that certain things do not have brands or series, e.g.
a film is modelled as a one-off episode.

This leads to some difficult questions for a search application. If a user
searches for the text ``doctor who'', are they expecting a link to the
latest episode to watch on iPlayer, information about the next episode
-- such as when it is due to be broadcast -- or a link to the overall
home page for the entire Doctor Who brand?

The model also skims the surface of the sport ontology \cite{rayfield2011bbc}
created before the 2012 Olympic games, which aims to model the whole domain
of sporting personalities, events and competitions (and more). This might
be too fine-grained for the domain model used within the search application,
but it is likely that people will want to search for competitions like
``World Cup'' or sporting disciplines such as ``football''. A search application
that understands these concepts as entities in their own right may well
be able to direct users at a curated, dedicated ``home page'' thereof alongside
simply matching articles and other works that contain those terms only.

\begin{comment}
@startuml model.png

skinparam monochrome true
skinparam circledCharacterRadius 0
skinparam circledCharacterFontSize 0


package CreativeWorks {
class CreativeWork {
  title : String
  dateModified : DateTime
  dateCreated : DateTime
  category : String
  description : String
  thumbnail: Image
}
class NewsItem extends CreativeWork
class BlogPost extends CreativeWork
class LiveCoverage extends CreativeWork
class LiveEventPage extends CreativeWork

interface Format
class TextualFormat extends Format
class VideoFormat extends Format
class InteractiveFormat extends Format
class ImageFormat extends Format
class AudioFormat extends Format
class PictureGalleryFormat extends Format
}

package Programmes {
class Programme
Programme <|-r- CreativeWork
class Brand extends Programme
class Series extends Programme
class Episode extends Programme

class Service
class Version
class Broadcast
class Ondemand
}

Service "1" -- "0..n" Programme : masterBrand
Episode "1" *-- "0..n" Version : hasVersions

Version "1" -- "0..n" Broadcast : broadcasts
Version "1" -- "0..n" Ondemand : availableOndemandWindows

Broadcast "1..n" -- "1" Service : broadcastOn
Ondemand "1..n" -- "1" Service : availableOn

Brand "1" *-r- "0..n" Series
Series "1" *-d- "0..n" Series
Series "1" *-r- "0..n" Episode

package Concepts {
class WebDocument
class Thing {
  preferredLabel
}
class Person extends Thing
class Event extends Thing
class Place extends Thing
class Theme extends Thing
class Organisation extends Thing
}

CreativeWork "1" -- "1..2" WebDocument : primaryContent
Thing "1" -- "0..n" WebDocument : primaryTopic
CreativeWork "0..n" -- "1" Format : primaryFormat
CreativeWork "0..n" -- "0..n" Thing : tag
Thing "0..n" -- "0..n" Thing : notablyAssociatedWith


package Sport {
class SportCompetition extends Event
class SportDiscipline extends Thing
}

Event "0..n" -- "0..n" Programme : coverage
SportCompetition "0..n" -- "0..n" Person : competesIn
SportDiscipline "1" -- "0..n" SportCompetition

@enduml
\end{comment}
\begin{sidewaysfigure}
  \begin{center}
    \includegraphics[width=\linewidth]{model.png}
  \end{center}
  \caption{Domain model for content items and other things pertinent to BBC contenet\label{model}}
\end{sidewaysfigure}

\chapter{Analysis}
\label{analysis}

\chapter{Evaluation}

\appendix
\chapter{Proposal}

\section{Introduction}

This proposal is for coursework describing the design of a search application
that allows visitors to the BBC website to find anything the BBC offers
online. The focus of the design is likely around the content indexing
systems, which need to support requirements from all areas of content
production.

The scope of the coursework will be restricted to a subset of the content
areas of the BBC website so as to focus on the differing challenges, but
ensure the exercise is concise and free from unnecessary complexity.

\section{The Problem}

Historically, the BBC has grown its presence on the World Wide Web by
allowing individual areas within the organisation to manage their own web
pages independently. This has led to a large variation in the ways
information is stored and how that content is formatted and is accessed.

This historic, technical variation adds to the existing complexity behind
the variation in the nature of the content as the web has matured. Traditional
searches that are optimised for textual items such as news articles might
not work so well for indexing online games aimed at children. Web products
such as iPlayer offer video and audio content with very little accompanying
text.

Website visitors are likely to perceive the BBC website as a single entity
and a unified search can meet those expectations by providing results
across all content areas for any search. There are many
challenges in providing a query interface that can serve varying
mixtures of types of content appropriately for different searches.

There is a dual need to index both a back catalogue of already-published
items and new content as it is written as timely as possible. Some areas
-- such as news -- will require very up-to-the-minute changes as they happen.
This need becomes even stronger for news and sport events that are being
covered live with continuous updates.

Finally, there are areas of content on the BBC website that no longer have
active teams maintaining them. When these areas are just pages lying on
a disk on a server, there is little option other than crawling the live
website to see what is still available.

\section{Proposed System}

The system to be designed is an application with interfaces for sending
and later retrieving content items based on searches given by website users.

The existence of a generic search engine capable of text analysis, indexing
and retrieval can be assumed. Systems that perform these tasks are
already available as free software or proprietary offerings. The bulk of
the problem is in the provision of an application interface that can
accept a large variety of content and interface with content production
workflows with minimal effort required of other teams.

Ultimately, the system needs to focus on the needs for the web application
that will be built to meet users' needs. The requirements
for this front end application are largely speculative and hypothetical
at this stage however.


\bibliographystyle{cell}
\bibliography{bibtex}

\end{document}
