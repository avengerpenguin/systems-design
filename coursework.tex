\documentclass[a4paper]{report}

\title{Designing a Search for the BBC}
\author{Ross Fenning}

\begin{document}

\maketitle

\tableofcontents

\chapter{The Problem}

\section{Rationale}

The BBC has been publishing content on the World Wide Web since the
mid 1990s and since then the amount and the diversity has increased
exponentially. Large websites or indeed the web as a whole would not have
been usable nor useful without the rise in quality of web search engines to
help people find content based on keywords or phases that describe or
appear in the sought after pages.

A large challenge for any web search application is to provide a common
interface and set of user interactions that can equally index, search
and link to a diverse range of types of information -- be it in the form
of text, images, video or games. A more recent challenge has been to
achieve this in a near-real time way to catch up with the rapid rate at
which content is added to the web (particulary from microblogging websites
such as Twitter).

Whilst there are successful general web search engines such a Google that
will find any piece of content, there is a perceived need for a
BBC-specific search that makes better use of the internal knowledge the BBC
has about its own content.

\section{Objectives}

The purpose of this report is to start the high-level design -- and perhaps
look at the lower levels of one or two aspects -- of a search application
specific to BBC content. The intent is to provide a consistent user interface
that allows the audience to find equally news articles, sport results,
TV catch-up, education resources and everything else the BBC produces.

The target audience for the BBC is effectively the entire population of the UK
and amongst those that do make use of BBC services, there is much diversity
of needs, preferences and technical ability. It is clear it is no small
task to design a search-based discovery mechanism of millions of diverse
pieces of content aimed at millions of diverse people.

A suitable approach for tackling real-word problems is \emph{Soft Systems Methodology} (SSM)
\cite{checkland2006learning}, which will allows us to stand back from
an ontological approach of defining what the search system \emph{is} or
\emph{comprises} and instead take an \emph{epistemological} view of search
as a system. With this view, we could consider a system that holistically
transforms members of the public's desires to find online content into
the consumption of that content -- whether those desires are \emph{precise}
(e.g. they want an exact article known by headline they saw earlier or a
particular programme they missed on television) or those desires are
\emph{fuzzy} (e.g. news about a certain topic, any comedy programme, learning
materials about the Industrial Revolution).

Checkland \cite{checkland1990soft} decribed a \emph{Rich Picture} approach to
representing a problem situation early in SSM approaches.
Given the size and complexity of the
search system as a whole, a useful initial step is to create such an informal
representation of what is known about the problem. Figure~\ref{rich-picture}
shows what I know of the audience, search and most BBC online content areas.
Note that not all areas are covered and a strong emphasis is placed on TV
catch-up (e.g. via the iPlayer product). Radio catch-up is not mentioned
as it shares a lot of similarity with television in terms of use and any
differences are out of scope for this design.

\chapter{Design}

\section{Use cases}

\end{comment}
\begin{figure}[p]
  \begin{center}
    \includegraphics[width=\linewidth]{use_case.png}
  \end{center}
  \caption{Use case diagram for BBC Search application\label{use-case}}
\end{figure}

\chapter{Analysis}

\chapter{Evaluation}

\appendix
\chapter{Proposal}

\section{Introduction}

This proposal is for coursework describing the design of a search application
that allows visitors to the BBC website to find anything the BBC offers
online. The focus of the design is likely around the content indexing
systems, which need to support requirements from all areas of content
production.

The scope of the coursework will be restricted to a subset of the content
areas of the BBC website so as to focus on the differing challenges, but
ensure the exercise is concise and free from unnecessary complexity.

\section{The Problem}

Historically, the BBC has grown its presence on the World Wide Web by
allowing individual areas within the organisation to manage their own web
pages independently. This has led to a large variation in the ways
information is stored and how that content is formatted and is accessed.

This historic, technical variation adds to the existing complexity behind
the variation in the nature of the content as the web has matured. Traditional
searches that are optimised for textual items such as news articles might
not work so well for indexing online games aimed at children. Web products
such as iPlayer offer video and audio content with very little accompanying
text.

Website visitors are likely to perceive the BBC website as a single entity
and a unified search can meet those expectations by providing results
across all content areas for any search. There are many
challenges in providing a query interface that can serve varying
mixtures of types of content appropriately for different searches.

There is a dual need to index both a back catalogue of already-published
items and new content as it is written as timely as possible. Some areas
-- such as news -- will require very up-to-the-minute changes as they happen.
This need becomes even stronger for news and sport events that are being
covered live with continuous updates.

Finally, there are areas of content on the BBC website that no longer have
active teams maintaining them. When these areas are just pages lying on
a disk on a server, there is little option other than crawling the live
website to see what is still available.

\section{Proposed System}

The system to be designed is an application with interfaces for sending
and later retrieving content items based on searches given by website users.

The existence of a generic search engine capable of text analysis, indexing
and retrieval can be assumed. Systems that perform these tasks are
already available as free software or proprietary offerings. The bulk of
the problem is in the provision of an application interface that can
accept a large variety of content and interface with content production
workflows with minimal effort required of other teams.

Ultimately, the system needs to focus on the needs for the web application
that will be built to meet users' needs. The requirements
for this front end application are largely speculative and hypothetical
at this stage however.


\bibliographystyle{cell}
\bibliography{bibtex}

\end{document}
