\section{Introduction}

This proposal is for coursework describing the design of a search application
that allows visitors to the BBC website to find anything the BBC offers
online. The focus of the design is likely around the content indexing
systems, which need to support requirements from all areas of content
production.

The scope of the coursework will be restricted to a subset of the content
areas of the BBC website so as to focus on the differing challenges, but
ensure the exercise is concise and free from unnecessary complexity.

\section{The Problem}

Historically, the BBC has grown its presence on the World Wide Web by
allowing individual areas within the organisation to manage their own web
pages independently. This has led to a large variation in the ways
information is stored and how that content is formatted and is accessed.

This historic, technical variation adds to the existing complexity behind
the variation in the nature of the content as the web has matured. Traditional
searches that are optimised for textual items such as news articles might
not work so well for indexing online games aimed at children. Web products
such as iPlayer offer video and audio content with very little accompanying
text.

Website visitors are likely to perceive the BBC website as a single entity
and a unified search can meet those expectations by providing results
across all content areas for any search. There are many
challenges in providing a query interface that can serve varying
mixtures of types of content appropriately for different searches.

There is a dual need to index both a back catalogue of already-published
items and new content as it is written as timely as possible. Some areas
-- such as news -- will require very up-to-the-minute changes as they happen.
This need becomes even stronger for news and sport events that are being
covered live with continuous updates.

Finally, there are areas of content on the BBC website that no longer have
active teams maintaining them. When these areas are just pages lying on
a disk on a server, there is little option other than crawling the live
website to see what is still available.

\section{Proposed System}

The system to be designed is an application with interfaces for sending
and later retrieving content items based on searches given by website users.

The existence of a generic search engine capable of text analysis, indexing
and retrieval can be assumed. Systems that perform these tasks are
already available as free software or proprietary offerings. The bulk of
the problem is in the provision of an application interface that can
accept a large variety of content and interface with content production
workflows with minimal effort required of other teams.

Ultimately, the system needs to focus on the needs for the web application
that will be built to meet users' needs. The requirements
for this front end application are largely speculative and hypothetical
at this stage however.
